\documentclass[12pt]{article}
%\linespread{1.6}
\usepackage{setspace}
\doublespacing
\usepackage[hmargin=1in,vmargin=1in]{geometry}                % See geometry.pdf to learn the layout options. There are lots.
\geometry{letterpaper}                   % ... or a4paper or a5paper or ... 
%\geometry{landscape}                % Activate for for rotated page geometry
%\usepackage[parfill]{parskip}    % Activate to begin paragraphs with an empty line rather than an indent
\usepackage{graphicx}
\usepackage{amssymb}
\usepackage{amsfonts}
\usepackage{epstopdf}
\usepackage{tikz-timing}
\usepackage{multirow}
\usepackage{here}
\usepackage{flafter}


\newcommand{\HRule}{\rule{\linewidth}{0.5mm}}
\newcommand{\superscript}[1]{\ensuremath{^{\textrm{#1}}}}
\newcommand{\subscript}[1]{\ensuremath{_{\textrm{#1}}}}


\title{EE3102 Design Report} %Metadata
\author{Patrick Barrett}
\date{}                                           % Activate to display a given date or no date

\begin{document}

\begin{singlespace}

\begin{center}
 
 

 
\textsc{\LARGE EE3102}\\[0.5cm]
 
%\textsc{\Large Stuff}\\[0.5cm]

\textsc{\Large 9 March 2012}\\[0.5cm]

\textsc{\Large University of Minnesota}\\[0.5cm]
 
 
% Title
\HRule \\[0.4cm]
{ \huge \bfseries Design Report}\\
 
\HRule \\[1cm]
 
% Author and supervisor
\begin{minipage}{0.4\textwidth}
\begin{flushleft} \large
\emph{Project Members:}\\
Patrick \textsc{Barrett} \\
Mark \textsc{Pennebecker} \\
Kayiita  \textsc{Johnson} \\
Prashant \textsc{Dhakal} \\
\end{flushleft}
\end{minipage}
\begin{minipage}{0.4\textwidth}
\begin{flushright} \large
\emph{Submitted to:} \\
Professor \textsc{Higman}\\
\end{flushright}
\end{minipage}
\end{center}

\vfill

\section*{Abstract}
The general purpose of this project is the design and construction of a music tuner. Specifically, this
tuner will be checking whether or not an input pitch matches 440Hz (A). The input will be read from
either a microphone or a coax cable fed from an oscilloscope. The purpose of a tuner is not only to test
on whether or not a pitch matches a note, but also to see if the input pitch is sharp or flat (whether the
pitch is above or below the given pitch). Our tuner will indicate this information through a series of
LEDs. We plan to use a microcontroller to analyze the input and control the output LEDs.
\end{singlespace}


\newpage
\section{Introduction}


The methodology for finding pitch in our tuner is largely based upon the theory behind musical tones
and how those pitches are measured.

The pitch we are measuring is called A440, which is a standard tuning pitch for musical instruments.
The 440 denotes the frequency in Hz of the pitch, which is called A by convention. An octave is defined
as a ratio of 2:1 between 2 pitch frequencies. In this case, the pitch A which is an octave above has a
frequency of 880 Hz, while the pitch A below the standard A has a frequency of 220Hz. Each octave
is split into 12 'semitones', which are considered distinct pitches within the octave. Since we are
trying to determine if the input is in tune (matching the 440Hz frequency) we require further precision
past the semitone measurement. Thus, we need to measure the accuracy of the input pitch using the
unit known as the cent. A cent is a further division of the interval between 2 semitones. The interval
between 2 semitones is divided into 100 cents. This means that each octave is divided into a total of
1200 cents and that the cent represents a different range of frequency depending on the octave.

We were tasked with measuring the difference between the input pitch and A440 in terms of 5 cent
intervals. This specification of the interval of measurement was given by the project requirements, but
it does make sense given that there is little audible difference between adjacent cent frequencies.

\section{Overview}
The basic flow of the signal is shown in Figure \ref{block_overview}. After the input is received from the function 
generator (microphone input will need a few more steps), the signal will pass through a bandpass filter to help
remove any noise from the signal. After being filtered, the signal will pass through a preamp, to boost
the signal so that it can be further processed by the later stages. The signal is then sent to a square
wave converter, which will take the analog signal and convert it into a series of square pulses to be
handled by the microcontroller. The microcontroller
will take the signal generated from this stage and measure the frequency of the square pulses over a certain
period of time. Using this information, the microcontroller will choose light up an LED array to display how close the
signal is to the desired pitch.

\begin{figure}[H]
\centering
	\includegraphics[width=6in]{"Block Overview"}
	\caption{Block Overview}
	\label{block_overview} 
\end{figure}

\section{Analog Subcircuit}
The analog portion of the circuit is a single IC consisting of four operational amplifiers, three of which are used. See Figure \ref{analog_block} for overview of the usages, where each stage uses a single opamp. The Pre-Amp is simply a voltage follower with a blocking capacitor and a voltage divider on the input to reference the incoming signal to 1/2 $V_{CC}$. The Filter is a simple inverting amplifier configuration with filtering caps which creates a band pass filter. This was designed to have an $f_{cH }= 880 \mbox{Hz}$ and an $f_{cL} = 220 \mbox{Hz}$. Those frequencies being an Octive above and below the center frequency respectively. The Square Wave Generator is simply configured as a comparator refrenced to 1/2 $V_{CC}$. The output of the comparator is then fed to the microcontroller to be processed further and interpreted.

\begin{figure}[H]
\centering
	\includegraphics[width=6in]{"Analog Subcircuit"}
	\caption{Block Diagram of Analog SubCircuit}
	\label{analog_block} 
\end{figure}

\section{Timing}

In order to calculate the value of a specific cent frequency, the following equation was used:

\begin{equation}
b = a * 2^{n \over 1200}
\end{equation}
%Switch 'lower' and 'freq' - I disagree.
where b is the desired frequency, a is a lower frequency than b, and n is the number of cents in between
a and b. For the positive cent range, we used a = 440Hz and the above formula where n was the number
of cents above 440Hz. For the negative cent range, a = 220Hz and n was calculated from

\begin{equation}
n = 1200 - |\mbox{cent offset}|
\end{equation}

so if we wanted to find the frequency of -5 cents from A440, n would be 1195. The frequencies found
from this formula can be seen in Table \ref{Freq_Table}. Knowing the frequencies corresponding to the cent ranges
is useful, but they must be manipulated somewhat for the microcontroller to interpret. The CCP module
on the microcontroller measures events by using an internal timer which increments on the instruction
clock (Fosc / 4). Therefore, the frequencies calculated in the previous stage need to be converted into a
number of instruction cycles. The following equation describes this calculation:

\begin{equation}
(\mbox{cent frequency})^{-1} * \left({ \mbox{Instruction Cycles} \over \mbox{Second}}\right)
\end{equation}

The converted values calculated from this equation can also be found in Table \ref{Freq_Table}. The Fosc, which is the
internal clock frequency, is 16MHz. This means that there are $4 *106$ instructions per second. We could
increase the resolution of our measurements by increasing the clock speed of the microcontroller, but
we felt that keeping our design low powered was more important than the increase in resolution gained
by using a higher voltage on the microcontroller.

\begin{table}
   \centering
    \begin{tabular}{|r|l|l|l|l|}
        \hline
        Cents from & Frequency& Cycles @ & Lower & Upper \\
        A440 & (Hz) & Fosc = 16MHz & Cycle Bound & Cycle Bound \\ \hline
       -35 & 431.1939264 & 9276 & 9263 & TMR \\ \hline
       -30 & 432.4410634 & 9250 & 9236 & 9263 \\ \hline
       -25 & 433.6918074 & 9223 & 9210 & 9236 \\ \hline
       -20 & 434.946169 & 9197 & 9184 & 9210 \\ \hline
       -15 & 436.2041585 & 9170 & 9157 & 9184 \\ \hline
       -10 & 437.4657865 & 9144 & 9130 & 9157 \\ \hline
       -5 & 438.7310635 & 9117 & 9104 & 9130 \\ \hline
       0 & 440 & 9091 & 9078 & 9104 \\ \hline
       5 & 441.2726067 & 9065 & 9052 & 9078 \\ \hline
       10 & 442.588941 & 9038 & 9025 & 9052 \\ \hline
       15 & 443.8288729 & 9012 & 8999 & 9025 \\ \hline
       20 & 445.1125537 & 8986 & 8974 & 8999 \\ \hline
       25 & 446.399944 & 8961 & 8948 & 8974 \\ \hline
       30 & 447.6910645 & 8935 & 8922 & 8948 \\ \hline
       35 & 448.985916 & 8909 & 1 & 8922 \\ \hline
    \end{tabular}
	\caption{Cent Step Calculations}
	\label{Freq_Table} 
\end{table}

The lower and upper bounds were calculated by averaging adjacent cycle counts. It is important to
note that the cycle counts are whole numbers since it is not possible to have a fractional cycle. All cycle
values were rounded to the nearest integer value, which means that the ranges intervals may be off by
at most 1-2 cycles, which we deemed as an acceptable imprecision.



Our tuner will take an input signal in the form of a square wave. to be tested. When the input pitch matches A-440Hz, the
center LED will light up. As the pitch moves out of tune, other LEDs will light up to signify how sharp or
flat the pitch is compared to A. These LED's will be set to 5 cent intervals ranging from -30 to 30 cents
around A. Anything above or below these ranges will simply light up the leftmost or rightmost LED
depending on whether they are above or below the required pitch. The decision of which LED to light up
will be handled by a microcontroller.



\section{Microcontroller}
We decided to use a microcontroller for the signal processing element of our tuner. Microcontrollers
allow quite a bit of flexibility in the processing of the audio signal. We chose the PIC18f4550 partially
because this microcontroller had enough ports to drive our LED array. However, the primary reason
that this microcontroller was chosen was due to our prior familiarity with the architecture and
functionality of the device. Given the limited requirements of the tuner device, there were no special
requirements of the microcontroller, which gave us a great degree of flexibility with our choice.

The code for the microcontroller is reliant on three main functionalities of the PIC18f4550: the TMR1
module, the CCP module, and the Primary Idle Mode.

The first module we used is called the TMR1 module, which is a simple timer module. This timer was
configured to increment on each instruction cycle to ensure the highest resolution possible. This timer
was chosen because it had the ability to hold 16 bit numbers, which was necessary for the number of
cycles we needed to count. This timer also works in conjunction with the CCP module, which will be
discussed in greater detail later. We used the timer to count the number of cycles for each period of the
input signal, but it also has another important function. If the timer were to roll over (meaning that it
attempted to increment above $2^16$), then the program will vector to an interrupt and reset the averaging
function used in the CCP interrupt. If this event occurred, then that would mean that the input signal
was no longer transmitting, meaning that it would be necessary to clear out the stored averaging values
in order to maintain a clear reading the next time an input was detected on the port. More information on
the TMR1 module can be found on page 143 of the PIC1f4550 datasheet.

The CCP (Capture, Control , Pulse Width Modulation) Module has a wide variety of uses. For
this application, only the capture mode will be used. While configured in the capture mode, the
microcontroller waits for a trigger on an event occurring on a specific port of the device. This event, which
the user can define as a falling or rising edge transition, will generate an interrupt and vector the CPU of
the microcontroller away from the main function. We chose to configure the capture module to copy
the contents of the TMR1 counter register into the CCP1REG (a register for use by the CCP module).
This transfer is performed automatically by the microcontroller. Basically, this interrupt is generated for every complete
cycle of the input signal, meaning that the timer module will record the number of instruction cycles
for each period of the signal. The cycle counts corresponding to the cent ranges are shown in Table
\ref{Freq_Table}. The cycle count is processed within the interrupt and averaged to ensure that the measurements
accurately represent the frequency of the input signal. More information on the CCP modules can be found
on page 129 of the PIC18f4550 datasheet.

The final module that our code utilizes is called the Primary Idle Mode. Since the majority of time
our code is spent waiting on interrupts to be generated, much of the microcontroller can be disabled
during the idle periods. The primary idle is a configuration of the sleep mode available for use on
the PIC18f4550. This configuration disables the CPU of the microcontroller, meaning that no more
instructions will be processed. Although the CPU is disabled, the peripheral devices, such as the
timer modules, will continue to be clocked from the primary clock source. There is a bit of a delay
in restarting the microcontroller from the idle mode, but since the TMR1 module is copied over to
the CCP1REG by internal hardware, this does not affect the accuracy of our measurements. Also, since our
instruction clock was on the order of about 3dB faster than an octave above the desired pitch frequency,
we assumed that the wake up time was fairly negligible. We made the same general assumption about
the interrupt service routine which calculates the average value. In other words, given the fact that we
will be executing approximately 9000 cycles for the desired pitch range, our ISR will not overlap 2
sequential CCP events. More information on the Idle modes can be found on page 40 of the 
PIC18f4550 datasheet.

Figure \ref{micro_chart}  represents the basic flow of the microcontroller program.  As can be seen from the figure, after the setup and main functions have been run, the microcontroller enters an idle mode.  This idle mode is exited when an interrupt is generated.  If the interrupt is a CCP interrupt, the average cycle count will be updated.  This new average will then be used to decide which LEDs should be turned on.  The LEDs are update and program control returns to the main loop.  If a TMR1 interrupt is generated, both the average and the display will be reset, and program control will be returned to the main loop.

\begin{figure}[H]
\centering
	\includegraphics[height=3.5in]{"MicroChart"}
	\caption{Program Flow Diagram}
	\label{micro_chart} 
\end{figure}


\section{Power Consumption}
Power consumption was a big issue for this project because it is  meant to be run off of battery power. We will be powering the circuit from a rechargeable lithium polymer battery. Due to stock and size constraints we are limited to two options for capacity, 150 mAh and 3 Ah. Even if we assume the worst case calculations, see Table \ref{Power}, the 150 mAh battery will power the circuit for more than 3 hours. The real time will be closer to eight hours as we will be spending most of our time with the microcontroller sleeping and all the LEDs won't be constantly on.

\begin{table}[!h]
   \centering
    \begin{tabular}{|r|r|r|r|r|}
        \hline
         & Current Con.& Current Con.& Current Con. & Current Con. \\
        Device & (Worst Case)& (Typical, Active)& (Typical, Sleep) & (Typical, No Input) \\ \hline
       $\mu$C \footnotemark & 35 mA & 16 mA & 6.2 mA & 6.2 mA \\ \hline
       Analog Subcircuit \footnotemark & 680 $\mu$A & 400 $\mu$A & 400 $\mu$A & 400 $\mu$A \\ \hline
       LEDs \footnotemark & 7 mA & 4 mA & 4 mA & 1 mA \\ \hline
       Power Regulator \footnotemark & 50 $\mu$A & 50 $\mu$A & 50 $\mu$A & 50 $\mu$A \\ \hline 
       $Total$ & 42 mA & 20 mA & 11 mA & 8 mA\\ \hline 
    \end{tabular}
	\caption{Power Consumption}
	\label{Power} 
\end{table}

\footnotetext[1]{PIC18f4550 Datasheet Pages 375-376}
\footnotetext[2]{MCP6001 Datasheet}
\footnotetext[3]{${{3.3 V - 2.3 V} \over 1 K\Omega } = 1 \mbox{mA} *\mbox{ N number of LEDs}$}
\footnotetext[4]{MCP1703  Datasheet}

\section{Conclusion}
We will be using a Microchip PIC18 microcontroller with an analog filter setup to
construct this music tuner. Our coding structure uses the CCP module to interrupt at
certain intervals to calculate the cents of the input. We will first create a demo board
and then move towards constructing the final board for our final design. We will have some
difficulties when ascertaining the frequencies the LEDs actually light up for, and adjusting those values
to make sure it is accurate.

\section{Group Member Workload}

\begin{singlespace}
\begin{description}
  \item[Patrick Barrett] \hfill \\
	\begin{itemize}
	  \item Analog Circuit Design
	  \item Analog Simulation
	  \item Schmatic Capture
	  \item Parts Selection
	  \item PCB Layout
	  \item Design Report
	\end{itemize}
  \item[Mark Pennebecker] \hfill \\
	\begin{itemize}
	  \item Microcontroller Selection/Setup
	  \item Programming
	  \item Design Report
	\end{itemize}

  \item[Kayiita Johnson] \hfill \\
	\begin{itemize}
	  \item Programming
	\end{itemize}

  \item[Prashant Dhakal] \hfill \\
	\begin{itemize}
	  \item Simulation
	\end{itemize}

\end{description}
\end{singlespace}


\end{document}  