\documentclass[12pt]{article}
%\linespread{1.6}
\usepackage{setspace}
\doublespacing
\usepackage[hmargin=1in,vmargin=1in]{geometry}                % See geometry.pdf to learn the layout options. There are lots.
\geometry{letterpaper}                   % ... or a4paper or a5paper or ... 
%\geometry{landscape}                % Activate for for rotated page geometry
%\usepackage[parfill]{parskip}    % Activate to begin paragraphs with an empty line rather than an indent
\usepackage{graphicx}
\usepackage{amssymb}
\usepackage{amsfonts}
\usepackage{epstopdf}
\usepackage{tikz-timing}
\usepackage{multirow}
\usepackage{here}
\usepackage{flafter}

\newcommand{\HRule}{\rule{\linewidth}{0.5mm}}
\newcommand{\superscript}[1]{\ensuremath{^{\textrm{#1}}}}
\newcommand{\subscript}[1]{\ensuremath{_{\textrm{#1}}}}


\title{Lab 8:\\RC Step Response} %Metadata
\author{Patrick Barrett}
\date{}                                           % Activate to display a given date or no date

\begin{document}

\begin{singlespace}

\begin{center}
 
 

 
\textsc{\LARGE EE3102}\\[0.5cm]
 
%\textsc{\Large Stuff}\\[0.5cm]

\textsc{\Large 8 Febuary 2012}\\[0.5cm]

\textsc{\Large University of Minnesota}\\[0.5cm]
 
 
% Title
\HRule \\[0.4cm]
{ \huge \bfseries Step Response of an RC Circuit}\\[0.4cm]
 
\HRule \\[1cm]
 
% Author and supervisor
\begin{minipage}{0.4\textwidth}
\begin{flushleft} \large
\emph{Project Members:}\\
Patrick \textsc{Barrett} \\
Mark \textsc{Pennebecker} \\
Kayiita  \textsc{Johnson} \\
Prashant \textsc{Dhakal} \\
\end{flushleft}
\end{minipage}
\begin{minipage}{0.4\textwidth}
\begin{flushright} \large
\emph{Submitted to:} \\
Professor \textsc{Higman}\\
\end{flushright}
\end{minipage}
\end{center}


\section*{Abstract}
The general purpose of this project is the design and construction of a music tuner. Specifically, this
tuner will be checking whether or not an input pitch matches 440Hz (A). The input will be read from
either a microphone or a coax cable fed from an oscilloscope. The purpose of a tuner is not only to test
on whether or not a pitch matches a note, but also to see if the input pitch is sharp or flat (whether the
pitch is above or below the given pitch). Our tuner will indicate this information through a series of
LEDs. We plan to use a microcontroller to analyze the input and control the output LEDs.
\end{singlespace}

\section{Introduction}

Explanation of how cents and such work, pitches etc... (More to come on this later)

Our tuner will be accurate to ±5 cents for A-440Hz. When the input pitch matches this frequency, the
center LED will light up. As the pitch moves out of tune, other LEDs will light up to signify how sharp or
flat the pitch is compared to A. These LED's will be set to 5 cent intervals ranging from -30 to 30 cents
around A. Anything above or below these ranges will simply light up the leftmost or rightmost LED
depending on whether they are above or below the required pitch. The decision of which LED to light up
will be handled by a microcontroller.

\begin{figure}[H]
\centering
	\includegraphics[width=6in]{"Block Overview"}
	\caption{Block Overview}
	\label{1A} 
\end{figure}

The basic flow of the data will be as follows. After the input is received from the oscilloscope
(microphone input will need a few more steps), the signal will pass through a bandpass filter to help
remove any noise from the signal. After being filtered, the signal will pass through a preamp, to boost
the signal so that it can be further processed by the later stages. The signal is then sent to a square
wave converter, which will take the analog signal and convert it into a series of square pulses to be
handled by the microcontroller. The microcontroller will take the signal generated from this stage and
measure the frequency of the square pulses over a certain period of time. Using this information, the
microcontroller will choose light up an LED to display how close the signal is to the desired pitch.

\section{Body}

For the microcontroller aspect of the datapath, we decided to use a PIC18 device. Given the relative
simplicity of the work required by the microcontroller, we chose this device do to our familiarity with
the inner workings of the microcontroller rather than any special functionalities of the PIC18.

The CCP module of the PIC18 can be used to measure the time in clock cycles between events. In this

application we'll define an event as the falling edge of the square wave generated from the previous
stage. Using one of the internal timers of the PIC18 we'll be able to measure the time between events
and we can calculate the frequency from that time measurement. Since it is possible for the frequency
to fluctuate we will take an average over a (relatively) number of samples. Based upon the frequency
calculated, the microcontroller will turn on an LED corresponding with the calculated frequency. To
accomplish this, we plan on using a single pin per 5 cent range. (Note that we may end up using a shift
register to cut down on the number of pins required).

\section{Formatting Examples}

\begin{table}[H]
\centering
\caption{\label{measured_tau}Measured Values of $\tau$}
\begin{tabular}{|r|c|c|}
\hline
 			&9.879 K$\Omega$ (1)	&98.81 K$\Omega$ (2)	\\	\hline	
9.94 nF (A)		&1.460 mS	&2.596 mS	\\  \hline
103.5 nF (B)		&2.530 mS	&11.44 mS	\\  \hline
106.79 nF (Series)	&N/A		&1.750 mS	\\  \hline
8.64 nF (Parallel)	&N/A		&12.44 mS	\\  \hline
\end{tabular}
\end{table}
 Table \ref{measured_tau} is shown above.

%\newpage
\subsection{Appendix}
\subsubsection{Sample Calculations}
\begin{equation}
	v(t) = 6.080 + ( 0 - 6.080)e^{-t/98.20E-6}
\end{equation}
\begin{equation}
	v(t) = 6.080(1 - e^{-t/98.20E-6})
\end{equation}

%\subsection{Data}
%\begin{figure}[H]
%\centering
%	\input{fig1}
%	\caption{Voltage Across a Capacitor and Resistor in Parallel For 10K$\Omega$ and 0.01$\mu$F}
%	\label{1A} 
%\end{figure}


\end{document}  